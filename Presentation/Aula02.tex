\documentclass{beamer}

\usetheme{Boadilla}
\usepackage[utf8x]{inputenc}
\usepackage{default}
\usepackage{amsmath}
\usepackage{listings}

\usepackage{bussproofs}

\usepackage[nounderscore]{syntax}
\setlength{\grammarparsep}{2pt}
\setlength{\grammarindent}{2em}

\newcommand{\doubleplus}{+\kern-1.3ex+\kern0.8ex}
\newcommand{\type}[1]{\emph{#1}}        % Type names in the text.
\newcommand{\rname}[1]{\textbf{\scriptsize \emph{#1}}} % Rule names in the type system.
\newcommand{\trname}[1]{\emph{\Small\textbf{#1}}} % Rule names in the text



\usepackage{listings}

\lstdefinestyle{uSugar}{
  emph={define, lambda, syntax, loop, if, true, false},
  emphstyle=\bf,
  basicstyle=\footnotesize, showspaces=false,
  firstnumber=auto, numbers=left, numbersep=1pt
}

\lstdefinestyle{ast}{
  basicstyle=\footnotesize, showspaces=false,
  literate={->}{::=}{2}
}





\begin{document}

\frame{

\title{Diversão com sistemas de tipos}
\author{Elton M Cardoso}
\date{11/05/2020}

}

\section{Lang, uma linguagem imperativa}

\frame{ 
   \huge
   \begin{center}
      Lang !
   \end{center}
   \normalsize
}

\begin{frame}
   \frametitle{Recursos:}
   \begin{itemize}
    \item operadores aritméticos: + e -
    \item operadores Lógicos: \&\& (e lógico) e ! (negação)
    \item Comparações: $<$ e == .
    \item Comandos: if-then-else, while e atribuições.
    \item Funções podem ser definidas e chamadas a qualquer momento !
    \item Um programa será uma coleção de funções e a função principal deverá ser chamada de main  e não deve ter quaisquer argumentos !
    \item Não há comando explícito de retorno ! Toda função deve terminar com uma expressão e tal expressão é o resultado da função;
    \item Só pode-se operar entre valores de mesmo tipo !
    \item Int: (+,-,$<$,==); String (==, $<$), Bool (\&\&, ! , ==)
   \end{itemize}
\end{frame}

\begin{frame}[fragile]
   \frametitle{Sintaxe Abstrata}
  \begin{lstlisting}[style=ast,escapechar=@]
Prgram -> Func*
Func   -> Type ID Params Cmd  
Params -> Type ID (,Type ID)* | @$\lambda$@
Cmd -> Cmd ; Cmd
    | IF Expr Cmd Cmd
    | WHILE Expr Cmd
    | VAR = Expr
    | Expr
Expr -> Expr + Expr
      | Expr - Expr
      | Expr && Expr
      | Expr < Expr
      | Expr == Expr
      | ! Expr
      | ID | INT | true | false | STRING
Type -> Int | Bool | String | Void
    
\end{lstlisting}
\end{frame}

\begin{frame}[fragile]
   \frametitle{Exemplo de um programa em Lang - Sintaxe Concreta}
\begin{lstlisting}[style=ast,escapechar=@]
  Int sum(Int x){
      s = 1;
      while(0 < x){
        s = s + x;
        x = x - 1;
      }
      s;
  }
\end{lstlisting}
   \begin{itemize}
      \pause
      \item O tipo da função está correto ?
      \pause
      \item O Tipo da expressão x = x - 1 está correto ? 
      \pause
      \item Existe algum tipo para a variável s ? 
   \end{itemize}

\end{frame}

\begin{frame}[fragile]
   \frametitle{So many (dificult) questions !}

   \begin{itemize}
      \item O tipo da função está correto ? (\textcolor{red}{$Type Checking$})
      \item Para responder a essa pergunta usamos, assumimos que x tem tipo Int e verificamos que o retorno da função será de fato do tipo Int.
      \pause
      \item O Tipo da expressão x = x - 1 está correto ? (\textcolor{red}{$Type Checking$}) 
      \item Assumindo que x tem tipo Int, verificamos se o resultado do lado direito é compatível com a variável do lado esquerdo.
      \pause
      \item Existe algum tipo para a variável s ? (\textcolor{red}{$Type Inference$})
      \item O único tipo para s que faz com que o programa fique correto é Int.
   \end{itemize}

\end{frame}

\begin{frame}[fragile]
   \frametitle{Considere agora esse programa !}
\begin{lstlisting}[style=ast,escapechar=@]
  Int mistery(Int x){
      y = false;
      if(y){
         x = "nada";
      }
      else{
        x = x + 1; 
      }
      x;
  }
\end{lstlisting}
   \begin{itemize}
      \item O tipo da função está correto ?
      \pause
      \item Quando podemos considerar um programa correto ?  (\textcolor{red}{Para isso precisamos de uma especificação.}) 
      \pause
      \item O que é correto ?  (\textcolor{red}{Tudo aquido que atende a uma dada especificação pode ser dito correto em relação à especificação}) 
      \pause
      \item É sempre possível descobrir se um programa está correto ? (\textcolor{red}{Infelizmente não ! mas na dúvida diga que não ;-) } ) 
   \end{itemize}

\end{frame}


\begin{frame}[fragile]
   \frametitle{E quanto a este outro !}
\begin{lstlisting}[style=ast,escapechar=@]
  Bool mistery2(){
      z = z and !z;
      z;
  }
\end{lstlisting}

\end{frame}


\begin{frame}[fragile]
   \frametitle{As regras do jogo !}
  \begin{itemize}
   \item Para julgar se um programa, ou uma parte dele, está correto precisamos de um modelo de como um programa devia se comportar
   \item Em geral nos interessa que o programa não faça nada que possa colocar a máquina que o executa em um estado de erro !
   \item Além disso, só porque podemos , também proibimos certas operações as quais julgamos serem "maus habitos" na linguagem.
   \item Finalmente especificamos um conjunto de regras que dizem como um "bom programinha" deveria ser comportar
  \end{itemize}

\end{frame}

\begin{frame}[fragile]
   \frametitle{Sistemas de tipo- Formato das regras}
   
  \begin{tabular}{l}
    \AxiomC{Premissas }
    \RightLabel{\rname{Nome da regra} }
    \UnaryInfC{$ conclusao $} 
    \DisplayProof 
  \end{tabular}
 \begin{itemize}
   \item Há um ambiente de tipos $\Gamma$, um mapeamento de nomes para tipos que registra tipo de cada identificador (nomes de variáeis, de funções, de classes e etc) no programa.
   \item Dizemos que de $\Gamma$ podemos deduzir o um tipo de alguma coisa do seguinte modo: $\Gamma \vdash t:\tau$, onde $tau$ é um tipo qualquer e $t$ é um fragmento (termo) do programa.
 \end{itemize}

\end{frame}


\begin{frame}[fragile]
   \frametitle{Algumas regras para Lang !}
   
  \begin{tabular}{l}
    
    \AxiomC{ $(x,\tau) \in \Gamma$}
    \RightLabel{\rname{var} }
    \UnaryInfC{$ \Gamma \vdash var : \tau$} 
    \DisplayProof \\[1cm]
    
    \AxiomC{}
    \RightLabel{\rname{int} }
    \UnaryInfC{$ \Gamma \vdash INT : Int$} 
    \DisplayProof \\[1cm]
    
    \AxiomC{}
    \RightLabel{\rname{bool} }
    \UnaryInfC{$ \Gamma \vdash BOOL : Bool$} 
    \DisplayProof \\[1cm]
    
    \AxiomC{}
    \RightLabel{\rname{str} }
    \UnaryInfC{$ \Gamma \vdash STRING : String$} 
    \DisplayProof \\
  \end{tabular}

\end{frame}

\begin{frame}[fragile]
   \frametitle{Algumas regras para Expressões !}
   
  \begin{tabular}{l}
    
    \AxiomC{ $\Gamma \vdash x_1 : Int $}
    \AxiomC{ $\Gamma \vdash x_2 : Int $}
    \RightLabel{\rname{var} }
    \BinaryInfC{$ \Gamma \vdash x_1 + x_2 : Int$} 
    \DisplayProof \\[1cm]
    
    \AxiomC{ $\Gamma \vdash x_1 : Int $}
    \AxiomC{ $\Gamma \vdash x_2 : Int $}
    \RightLabel{\rname{var} }
    \BinaryInfC{$ \Gamma \vdash x_1 - x_2 : Int$} 
    \DisplayProof \\[1cm]
    
   \AxiomC{ $\Gamma \vdash x_1 : Bool $}
    \AxiomC{ $\Gamma \vdash x_2 : Bool $}
    \RightLabel{\rname{var} }
    \BinaryInfC{$ \Gamma \vdash x_1 \&\& x_2 : Bool$} 
    \DisplayProof \\[1cm]
     
    \AxiomC{ $\Gamma \vdash x_1 : Int $}
    \AxiomC{ $\Gamma \vdash x_2 : Int $}
    \RightLabel{\rname{var} }
    \BinaryInfC{$ \Gamma \vdash x_1 == x_2 : Bool$} 
    \DisplayProof \\[1cm]
   
  \end{tabular}

\end{frame}


\end{document}
